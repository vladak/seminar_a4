%\documentclass[article]{seminar}
%\documentclass[slidesonly]{seminar}
%\documentclass[a4paper]{seminar}

\usepackage{multicol}
\usepackage{tabularx}

\usepackage{nth}

%\input{seminar.bg2}
%\usepackage{seminar}

% color doc: http://en.wikibooks.org/wiki/LaTeX/Colors
% color defs: http://www.w3schools.com/html/html_colors.asp
\usepackage{color}

\definecolor{MyDarkBlue}{RGB}{71,57,146} % BlueViolet
\definecolor{MyDarkRed}{RGB}{139,00,00}


% hyperref docs:
%   http://www.tug.org/applications/hyperref/manual.html
%   http://en.wikibooks.org/wiki/LaTeX/Hyperlinks
%
\usepackage[bookmarks,breaklinks=true,%
	colorlinks=true,linkcolor=MyDarkBlue,urlcolor=MyDarkRed,%
	pdftex,%
	pdfauthor={Vladimir Kotal, Jan Pechanec},%
	pdftitle={Unix programming in C},%
	pdfsubject={Materials for the NSWI015 class at MFF-UK},%
	pdfkeywords={Unix, programming, C, MFF, MFF-UK},%
	pagebackref=true,%
	]{hyperref}
\usepackage{url}
% by default \url will use monospaced font. suppress this and use normal font.
\urlstyle{same}

% for strike-through (\sout)
\usepackage[normalem]{ulem}

% cannot use hyperref with this one !
%\usepackage{sem-a4}

\usepackage{alltt}
\usepackage{graphicx}
%\usepackage{psfrag}
%\usepackage{pstricks} % does not work with colortbl
%\usepackage{colortbl}

\usepackage{upquote}

%%% Macros %%%

% colored frame around source code path to examples
% which is actually a link
\newcommand{\example}[1]{\fcolorbox{cyan}{white}{\href{https://github.com/devnull-cz/unix-linux-prog-in-c-src/blob/master/#1}{\texttt{#1}}}}

% (janp)
\chardef\clqq=254  \sfcode254=0 \lccode254=0
\chardef\crqq=255  \sfcode255=0 \lccode255=0
%\DeclareRobustCommand\uv[1]{{\leavevmode\clqq#1\crqq}}
\DeclareRobustCommand\uv[1]{{\leavevmode{},,#1''}}
% (janp) end

\renewcommand{\slidelabel}{}
\setlength{\textwidth}{0.9\textwidth}

\newcommand{\sltitle}[1]{{\centering\textbf{\Large #1}
    \vskip 2em plus 0pt minus 2em\par}} % Slide title

% bold
\newcommand{\emsl}[1]{\textbf{#1}} % Emphasizing in slides


\newcommand{\emprg}[1]{\emph{\color[rgb]{1,0,0} #1}} % Emphasize in programs
\newcommand{\emblue}[1]{\emph{\color[rgb]{0,0,1} #1}} % emph in blue

% my very important note
\newcommand{\rednote}[1]{\color[rgb]{1,0,0} #1}

% What should be checked
\newcommand{\CHECK}[1]{{\color[rgb]{1,0,0} $\star$#1$\star$}}


\newenvironment{itemize2} % Itemize with smaller font
    {\begin{itemize}\small} {\end{itemize}}
    
\newsavebox{\boxTMP}
\newcommand{\raisetab}[1]{ % Align first table row with other text
    \sbox{\boxTMP}{\begin{tabular}{c}\hline X\\\hline\end{tabular}}
    \raisebox{\ht\boxTMP}{#1}}

\newcommand{\funnm}[1] {% Emphasized function name
    {\bf #1}}

\newcommand{\funml}[1] { % Multi-line function prototype
    \begin{minipage}{\slidewidth}
    \vspace{-1ex}\texttt{\begin{tabbing}#1\end{tabbing}}
    \end{minipage}}

\newcommand{\bs}{\char92\relax} % TT backslash
% http://tex.stackexchange.com/questions/9363/how-does-one-insert-a-backslash-or-a-tilde-into-latex
\newcommand{\tilda}{\char`\~} % TT tilda

\newtoks\prgcharsI\newtoks\prgcharsII
{\catcode`\_=13\catcode`\&=13\global\prgcharsI={_}\global\prgcharsII={&}}
\def\prgchars{ % Do not require backslashes for these characters often used
                % in C program source code
    \catcode`\_=13\catcode`\&=13
    \expandafter\def\the\prgcharsI{\_}\expandafter\def\the\prgcharsII{\&}}

\centerslidesfalse

\newcounter{topicval}

\newcommand{\slidecontents}[1]{%
  \setcounter{topicval}{#1}%
  \begin{itemize}
  \topic{1}{Introduction, Unix and C, programming tools}
  \topic{2}{Basic Unix concepts and conventions, its API}
  \topic{3}{Access rights, devices}
  \topic{4}{Process manipulation, program execution}
  \topic{5}{Signals}
  \topic{6}{Process synchronization and interprocess communication}
  \topic{7}{Network programming}
  \topic{8}{Programming with threads}
  \topic{9}{Appendix}
  \end{itemize}%
}
\newcommand{\topic}[2]{%
  \item\relax
  \ifnum #1=\value{topicval}%
    \textbf{#2}%
  \else
    #2%
  \fi
}

\begin{document}

\begin{slide}
\centerslidestrue
\begin{center}
\title{\LARGE Unix/Linux Programming in C}
\author{(NSWI015)}
\date{Version: \rm\today}
\maketitle

\vspace{2ex}
{\small (c) 2011 -- 2023 Vladim\'{i}r Kotal}\\
{\small (c) 2005 -- 2011, 2016 -- 2022 Jan Pechanec}\\
{\small (c) 1999 -- 2004 Martin Beran}

\vspace{2ex}
Department of SISAL\\
Faculty of Mathematics and Physics, Charles University\\
Malostransk\'{e} n\'{a}m. 25, 118 00 Praha 1

\end{center}
\end{slide}

\begin{itemize}
\item This is official material for the class \emph{Unix/Linux Programming in C}
(NSWI015) lectured at the Faculty of Mathematics and Physics, Charles University
in Prague.
\item This material is published under the
\href{http://creativecommons.org/licenses/by-nc-sa/3.0/cz/}{Creative Commons
BY-NC-SA 3.0} license and is always a work in progress, see the history on
GitHub:\\
\url{https://github.com/devnull-cz/unix-linux-prog-in-c}
\item To download the latest version, go to the
\href{https://github.com/devnull-cz/unix-linux-prog-in-c/releases}{releases}
on GitHub.
\item Source code referenced from this material is published in
\href{http://creativecommons.org/licenses/publicdomain/}{Public Domain} unless
specified otherwise in the files.
\item The source code files can be found on GitHub here:\\
\url{https://github.com/devnull-cz/unix-linux-prog-in-c-src}
\item In case you find any errors either in the text or in the example programs,
we would appreciate you letting us know. Especially do not hesitate to create new
issues on \url{https://github.com/devnull-cz/unix-linux-prog-in-c/issues}.
\end{itemize}

\pagebreak

\begin{slide}
\sltitle{Contents}
\slidecontents{0}
\end{slide}

\begin{itemize}
\item This lecture is mostly about Unix principles and Unix programming in the~C
language.
\item \emsl{The lecture is mostly about system calls, i.e. an interface between a
user space and system kernel.}
\item For the API, we will follow the \emph{Single UNIX Specification,
version~4} (SUSv4). Systems that submit to the Open Group for certification and
pass conformance tests are termed to be compliant with the UNIX standard
UNIX~V7.  Some versions of AIX, HP-UX and macOS on selected architectures
are compliant with the previous version SUSv3
(\url{http://www.opengroup.org/openbrand/register/xy.htm}).
\item The specific source code examples linked from this material are usually
tested on Solaris, macOS and Linux.
\end{itemize}

\endinput


\begin{slide}
\centerslidestrue
\sltitle{The End.}
\end{slide}

\end{document}
